%%%%%%%%%%%%%%%%%%%%%%%%%%%%%%%%%%%%%%%%%%%%%%%%%%%%%%%%%%%%%%%%%%%%%%%%%%%%%%%%%%%%%%%%%%%%%%%%%%%%%%%%%%%
%%%%%%%%%% Enables support of track changes, marking of draft text and review comments %%%%%%%%%%%%%%%%%%%%
%%%%%%%%%%%%%%%%%%%%%%%%%%%%%%%%%%%%%%%%%%%%%%%%%%%%%%%%%%%%%%%%%%%%%%%%%%%%%%%%%%%%%%%%%%%%%%%%%%%%%%%%%%%

% Author Jose Teixeira <japolinario@gmail.com> 20 Oct 2016


\usepackage{xargs} 
%\usepackage[dvipsnames]{xcolor}

\usepackage{stackengine}

% 
%\RequirePackage[]{todonotes}
\RequirePackage[obeyDraft]{todonotes} % with obay draft todonotes apear only in draft mode
%\usepackage[]{todonotes}


%%%%%%%%%%%%%%%% Based on the todonotes packacke  %%%%%%%%%%%%%%%%%%%%%%%%%%%%

%The pack­age pro­vides ex­tended ver­sions of \new­com­mand and re­lated LaTeX com­mands, which al­low easy and ro­bust def­i­ni­tion of macros with many %op­tional ar­gu­ments, us­ing a clear and sim­ple xkey­val-style syn­tax.


% %The pack­age lets the user mark things to do later, in a sim­ple and vi­su­ally ap­peal­ing way
% \usepackage[colorinlistoftodos,prependcaption,textsize=tiny]{todonotes}
% 
% % Uncomment to disable the todo lists and marks - 
% %\usepackage[disable, colorinlistoftodos,prependcaption,textsize=tiny]{todonotes}

%% It a good practice to couple this with a config.tex file 

\newcommandx{\error}[2][1=]{\todo[linecolor=red,backgroundcolor=red!50,bordercolor=red,#1]{#2}}
\newcommandx{\unsure}[2][1=]{\todo[linecolor=red,backgroundcolor=red!25,bordercolor=red,#1]{#2}}
\newcommandx{\change}[2][1=]{\todo[linecolor=blue,backgroundcolor=blue!25,bordercolor=blue,#1]{#2}}
\newcommandx{\info}[2][1=]{\todo[linecolor=green,backgroundcolor=green!25,bordercolor=green,#1]{#2}}
\newcommandx{\improvement}[2][1=]{\todo[linecolor=green,backgroundcolor=green!25,bordercolor=green,#1]{#2}}
\newcommandx{\thiswillnotshow}[2][1=]{\todo[disable,#1]{#2}}

% This places the todonotes in the left
%\reversemarginpar

% Usage example I: \error{it's seems with two ee, not one e}{seems no sems}
% Usage example II: \improvement{To jose: Consider to find a good reference}{Swans are never black} 


%%%%%%%%%%%%%%%%%%%%%%%%%%%%%%%%%%%%%%%%%%%%%%%%%%%%%%%%%%%%%%%%%%%%%%%%%%%%%%

%%%%%%%%%%%%%%%% Simply  marking draft text in the violet color %%%%%%%%%%%%%%%%%%%%%%%%%%%%

% Environment to draft text 
\newenvironment{drafte} {\color{violet}}

% Usage example:  \begin{drafte} block of text in draft mode \end{drafte}                                 

%%%%%%%%%%%%%%%%%%%%%%%%%%%%%%%%%%%%%%%%%%%%%%%%%%%%%%%%%%%%%%%%%%%%%%%%%%%%%%%%%%%%%%%%%%%%
   
% Command to mark a paragraph as draft   
\newcommand{\draftp}[1]{\textcolor{violet}{#1}}  

% Usage example: \draftp{ This paragraph of text is still a drtaft }


%%%%%%%%% Based on discussion as https://tex.stackexchange.com/questions/142242/robust-way-to-mark-draft-text/142258 %%%%%%%%%%%

% Allows to make marks above and bellow the text -- to support review or warning authors. 

\setstackgap{L}{.5\baselineskip}
\newcommand\markabove[2]{{\sffamily\color{red}\hsmash{$\uparrow$}%
  \smash{\toplap{#1}{\scriptsize\bfseries#2}}}}
\newcommand\markbelow[2]{{\sffamily\color{red}\hsmash{$\downarrow$}%
  \smash{\bottomlap{#1}{\scriptsize\bfseries#2}}}}
  
% Usage examples I: \markbelow{l}{and lapped, I think} <--mark comments below to the left 
% Usage examples II: \markbelow{c}{and lapped, I think} <--mark  comments below centered over 
% Usage examples II: \markabove{r}{I'm sorry, but the links didn't show}%  <--mark  comments above to the right 
%%%%%%%%%%%%%%%%%%%%%%%%%%%%%%%%%%%%%%%%%%%%%%%%%%%%%%%%%%%%%%%%%%%%%%%%%%%%%%%%%%%%%%%%%%%%%%%%%%%%%%%%%%%%%%%%%%%%%%%%%%%%%%%%%%



%%%%%%%%%%% Box with Large Review comments 


% Command boc a given review    
\newcommand{\boxreview}[2]{\parbox{\linewidth}{
\vspace{0.2cm}
\scriptsize\bfseries\sffamily\color{red}{#1} commented -- {#2}}
\vspace{0.2cm}
}  

% Usage examples:  \boxreview{icis r1}{pointed out issues in the data collection}

%%%%%%%%%%%%%%%%%%%%%%%%%%%%%%%%%%%%%%%%%%%%%%%%%%%%%%% 




   
% Command to mark a paragraph as draft   
\newcommand{\newPara}[1]{\textcolor{Sepia}{#1}}  



\newenvironment{newStuff}%
%{\noindent\ignorespaces }%
%{\par\noindent%
%\ignorespacesafterend }
{
    \begin{color}{Sepia}
    \begin{tabular}{|p{1.0\textwidth}|}
    \hline\\
    }
    { 
    \\\\\hline
    \end{tabular} 
    \end{color}
    }
